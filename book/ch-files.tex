\chapter{Files, File Type, and File Type Identification}

In computer system engineering the term \emph{file} usually means a
sequence of zero or more bytes that can be saved in some kind of storage
device so that it can be repeatedly accessed. These bytes are known as the \emph{file contents}. 
Because at any given instance there is a specific number of bytes, every file inherently has a
\emph{file length}. 

There are other characteristics that we
commonly observe in files, such as file names, time stamps, owners,
\etc. Together these are sometimes called \emph{file
  properties} or \emph{file metadata}. \tabref{file-properties}
presents a list of file
properties seen on some operating systems. This list is not
exhaustive, and it is easy to find counter examples of files that lack any
(or all) of these properties. As a result, all that we
can say about files definitively is that they have zero or more bytes,
and that these bytes can be \emph{read} or \emph{written} (that is, they can be
copied from the file into the computer's memory, and new bytes can be
copied to the file). 


\begin{table}
\begin{tabularx}{\textwidth}{lX}
Property & Description \\
\hline
\\
\multicolumn{2}{l}{\textit{\small mandatory file properties:}}\\
File Contents & A sequence of bytes that makes up the file.\\
File Length   & The number of bytes in the \emph{file contents}.\\
\\
\multicolumn{2}{l}{\textit{\small optional file properties:}}\\
File Name & A sequence of 1 or more characters that can be used to reference the file. Files can have
zero, one, or multiple names.\\
Creation time (crtime) & The date and time when the file was created.\\
Modification time (mtime) & The date and time that the file's contents were last modified.\\
Access time (atime) & The date and time when a  byte of the file was last read.\\
Change time (ctime) & The date and time that the file's metadata was last modified. On Unix-based file systems
   the change time is frequently treated as if it were a creation time.\\
Inode number & Many file systems use a single integer to describe the location of a metadata block that provides a pointer to the file's contents. The block is called an \emph{inode} (index node).\\
Link Count (nlink) & On file systems that support \emph{hard links} (multiple names pointing at the same inode), the link count tracks the number of links.\\
Owner (uid) & The owner of a file is typically the operating system user that has the ability to change the file's properties. It is typically the creator of the file, although the owner can be changed.\\
Group (gid) & In addition to owners, most file systems allow a file to be put in one or more groups. Group membership can then be used to allow or prevent access to the file.\\
mode & Unix-based systems encode a ``mode'' in file system entries that are used to denote special files, such as those representing physical devices, FIFOs, sockets, and symbolic links.\\
Access Control Lists (ACLs) & An extended list of users and groups that are allowed (or disallowed) access to the file.
\end{tabularx}
\caption{Mandatory and optional file properties}\label{file-properties}
\end{table}

Files have been one of the primary abstractions provided by operating
systems for manipulating data for more than fifty years. Files are
the primary way that most computer information is stored, including:

\begin{itemize}
\item Text (typically as a sequence of bytes)
\item Digital photographs (\eg JPEG files)
\item Documents (\eg Microsoft Word Files, Adobe Acrobat files)
\item Programs (\eg source code and executables)
\item Databases (\eg Oracle database files, SQLite files)
\item Disk images (a byte-for-byte copy of a disk can be put into a file).
\end{itemize}

Because of their pervasive use, the identification, recovery and
manipulation of files is one of the primary tasks of computer
forensics. 

It's important to remember that files are just
\emph{abstractions}---there is nothing inherent in the design of
memory or storage systems that dictates the use of files. Indeed,
there are many examples on modern computer systems of information that
is typically \emph{not} stored in files on a running computer system, including::

\begin{itemize}
\item The contents of RAM, including system memory and video memory.
\item CPU registers, the page translation table, and the translation
  lookaside buffer (TLB).
\item BIOS ROM or Flash RAM used for booting the computer.
\item Boot blocks on the mass storage device.
\item Partitions used for virtual memory. (Operating systems such
  as Unix traditionally swapped to an unformatted portion of the disk
  called the \emph{swap partition}; it is only recently
  have operating systems started swapping to files.)
\item Partitions managed directly by a database. (Some databases
  directly access the raw storage to avoid the overhead of going
  through the file system.)
\end{itemize}

Of course, any of these pieces of information, like any other digital
data, can be extracted and stored in a file.

With the move to cloud-based services such as Facebook and Google
Docs, files are sure to become less important in computer forensics
and privacy auditing over the next decade. It is likely that storage
on clients such as mobile phones and laptops will increasingly be used
to cache information stored on a remote server, and the information
those servers will be stored in databases, rather than in files.

\section{File Type}

Before you can work with a file, it is necessary to know how the
information inside that file is organized. This organization is called
the file's \emph{type}. Privacy sensitive information can be missed if
a file's type is not properly identified, since many file types
require the use of special algorithms to interpret the data that they
contain. 

File types exist in an overlapping hierarchy, with many files having
dozens of types of varying levels of specificity. This complicates the
task of file type identification. For example, a program that is
compiled by a Java program may be thought of as a Java source file, a
source code file, or merely a text file: all are accurate descriptions
of the file's type.

Many file types are named after the specific version of a program. For
example, there is a Microsoft Word 2007 document file, which 
describes a particular format used by Microsoft Word to store
documents. Such names can be misleading, because popular file formats will
be adopted by many programs.

Desktop operating systems use a file's type to identify the program
that is run to ``open'' the file, as well as the icon that should be
used when the file is displayed in a windows.

\subsection{Identifying a File's Type}

Identifying file type is complicated not only because there are
hundreds of thousands of types, but because the standard for
identification is ultimately whether or not the file can be read and
its information extracted. 

When a file type is named after a program (as above), the name is
usually historical reference, since after a file type is released it
is common for developers to add support for this file type to other
programs. For example, Word 2007 files can be read by many different
programs. Adding to the confusion, different programs may display
different information given the same Word file.)

In general there are two ways to identify a file's type:

\begin{itemize}
\item Identification by filename extension or other Metadata.
\item Using content within the file.
\end{itemize}

\subsubsection{File Extension or other Metadata}

A file's \emph{extension} is the sequence of letters following the
last period (``.'') in the file's name. For example, the file
\emph{WINWORD.EXE} has ``EXE'' as an extension, while the file
\emph{WINWORD.TXT} has ``TXT'' as an extension.

Extensions were originally limited to three uppercase characters by
early computer systems. Modern systems allow extensions to be any mix
of letters, numbers and symbols that are valid for a file name. Some
systems treat extensions as case-sensitive, while others do not. That
is, on Windows the extensions ``.EXE'' and ``.exe'' are equivalent,
but on Unix the extensions ``.conf'' and ``.CONF'' are not.

Beyond the file's extension, operating systems can explicitly record
the file's type elsewhere in the file system.

Extensions and other metadata are not an accurate way to identify file type because
the extension is easily changed, either by accident or on
purpose. For example, changing the extension of a ``.java'' file to
``.txt'' on most desktop computers will change the way that the file's
icon is displayed as well as changing the program that runs when the
file is double-clicked, even though changing the name does not change
the file's contents.

\subsubsection{File Content}

The only reliable way to identify a file's type is by using the file's
content. These approaches are commonly used.

\begin{itemize}
\item \textbf{File Headers and Magic Numbers.} Many file types begin
  with a specific header, also called a ``magic number.'' For example,
  Windows executables begin with the sequence ``MZ'' (the initials of
  programmer Mark Zbikowski), ZIP files begin with the sequence ``PK''
  (the initials of programmer Phil Katz), and PDF files begin with the
  sequence ``\%!''.  Header recognition is a fast way to identify a
  file because it typically requires no more looking up a value in a
  table, but it can produce incorrect results because magic numbers are easily
  changed. 
\item \textbf{Internal Structure.} A second approach to identify a
  file's contents is by parsing the file's internal data structures
  and testing to see if they are internally consistent. This can be a
  computationally expensive approach, as each file type must be
  checked sequentially, but it is quite accurate.

\item \textbf{Statistical Modeling.} A third approach to file type
 identification is to compare the statistics of the file being
 identified to a corpus of files of known type. Although this is the
 least reliable approach for identifying files, it has the advantage
 that it can identify file fragments with much higher reliability than
 the other two approaches.
\end{itemize}

Ultimately, the gold standard of file type identification is to open a
file in an application program and see if the file's content can be
read. Unfortunately, this can be dangerous if there is a potential
that the files contain malware.

\subsection{Kinds of File Types}



Finally, some file
types are complex data structures that are stored by a specific
program and generally can't 

However
because of an accident of history, we also think of a file's type as
being the file's \emph{extension}---that is, the three or four
characters that come after the final period in the file's name.

For example, digital cameras will create files with names like
\emph{DSC0001.JPG} that contain digital images. Most people think of
these files as ``JPEGs'' because the names end with the letters ``.jpg'' or
``.jpeg''.  But what really makes a file a JPEG is
its conformance to a format dictated by the Joint Photographic Experts
Group (JPEG)'s 1992 standard, known formally as ISO/IEC IS 10918-1 and
as ITU-T Recommendation T.81. That format defines a ``chunk-based''
container file format that begins with the byte sequence |FF D8| and
ends with the byte sequence |FF D9|.

Likewise, files created by Microsoft's popular Word file have the
extension ``.doc'' or ``.docx''. Some people call these files ``DOC''
or ``DOCX'' files. In fact, files created by Word with the ``.doc''
extension are typically byte streams formatting according to Microsoft's Object Linking and
Embedding (MSOLE) format; look inside these files and you will see data
structures reminiscent of the FAT32 file system. Files with the ``.docx'' extension are
typically ZIP compression archives that contain structured XML that is
interpreted by Word to display an editable word processing
document. (However, encrypted ``.docx'' files follow a different
standard altogether.)
Microsoft Excel uses the same MSOLE and ZIP formats when saving files
with the ``.xls'' and ``.xlsx'' extensions, respectively.

In practice, the phrase \emph{file type} can mean one of several things:

\begin{itemize}
\item \textbf{The file's \emph{extension}.} Early computer systems
  required that file names be in the format of eight characters, a
  period, and three characters. The three characters after the period
  are called the extension and were used on some computers to denote
  the program that was used to write and read the file. Microsoft
  Windows allowed extensions to be any length and directly associated
  application programs with extensions through the Windows Registry;
  double-clicking on a file's icon caused the application program
  registered for that file extension to run and open the file.
\item \textbf{The file's Internet media type (also known its MIME
  type).} The internet media type is a sequence of letters that are
  transmitted with every byte stream is that is downloaded over the
  Internet using the HTTP protocol. JPEG files are downloaded with the
  Internet media type ``image/jpeg''. \wikipedia{http://en.wikipedia.org/wiki/Internet_media_type}. 
\item \textbf{The file's \emph{magic numbers} or \emph{file
    header}}. Both terms are used to describe a sequence of bytes at the
  beginning of the file that hold information about the file and may
  be used to identify it in some way.
\end{itemize}

Closely related to file type are:
\begin{itemize}
\item The file's \emph{header}.
\item The file's \emph{footer}.
\end{itemize}

Let us explore each of these in turn and see how they influence
file type and interpretation when performing computer forensics.

\subsection{File Extensions}

\emph{to be written.}

\subsection{Internet Media Types}

\emph{to be written}

\subsection{Magic Numbers}

The term \emph{magic number} is used to denote a constant byte or set
of bytes that typically appears at the beginning of a file. A common
magic number is the letters ``MZ'' denoting the start of a DOS or
Windows executable or Dynamic-Link Library (DLL). The SQLite database
uses the 15 letters ``SQLite format 3'' as its magic number. Magic numbers are
thus not really numbers, but actually sequences of characters. ``MZ'' has
the numeric value 23117 or 19802 depending on whether one is using
little-endian or big-endian math (\figref{calc-mz}). (Even though the
string ``SQLite format 3'' has a numeric value, it would never be used
in practice.)

\begin{figure}
\begin{Verbatim}
>>> ord('M') | ord('Z')<<8
23117
>>> hex(23117)
'0x5a4d'
>>> 
>>> ord('M')<<8 | ord('Z')
19802
>>> hex(19802)
'0x4d5a'
\end{Verbatim}
\caption{Computing the decimal and hex values of the character
  sequence ``MZ'' in little-endian and big-endian format with
  Python.}\label{calc-mz}
\end{figure}

While some file formats require magic numbers, others do not. Files in
the Hyper Text Markup Language originally were supposed to begin with
the string ``|<html|'' and nowadays should begin with the string
``|<!DOCTYPE|'', but most web browsers will do their best to display
any text as HTML if they detect HTML tags.

\subsection{File Headers}

\emph{To be written. Note that the file header may contain fields
  which contain metadata.}

\subsection{File Footers}

\section{Container Files}

\subsection{Chunk-based Container Files}


\subsubsection{GIF}
\emph{how GIF came about}

\subsubsection{TIFF}
\emph{What can be stored in a TIFF}

\subsubsection{JPEG}\label{jpeg-format}
\subsubsection{PNG}

\subsection{Directory-based Container Files}

\subsubsection{ZIP}

\subsubsection{PDF}

\subsection{Opportunities for Stegnagraphy}
The structure of these files means that there are many opportunities
to hide data in a container file that will be ignored by legitimate
software and processed by covert-communications programs.

\subsubsection{Stegnagraphy with Chunk-Based Files}
Create new chunk types that will be ignored.

Hide information at the end of each chunk.

Danger is that some programs may look for such data (or even error),
so it's important to test. It may be desirable to encrypt the data and
include cover data that gives the impression that the text is
legitimately placed there by a program that the analyst simply doesn't
have access to. For example, a new chunk type containing a string such
as ``gamma correction'' or ``color calibration.''

\subsubsection{Stegnagraphy with Directory-Based Files}

ZIP and PDF.

Store information in regions not associated with files.

Embed new file types that are ignored.

\subsubsection{Traditional Stegnography}
Finally, multi-media files can be used for traditional
stegnaraphy. For example, the bottom bits of a picture can be used to
encode another image....



\section{File Type Identification}

Because we do not have a really firm definition for ``file type,''
it's not clear what's meant by file type identification. Do we mean
finding an application that can ``open'' a file? Do we mean being able
to extract the content from a file so that it can be viewed/indexed/processed?

\subsection{Front-Reading vs. Rear-Reading}

\emph{JPEG files read from the front; ZIP files read from the back;
  this is a problem}


\subsection{The \emph{file} command and libmagic}

\emph{to be written}

\subsection{File Corruption}

\section{Content Extraction}

An alternative to file type identification is content extraction.

\subsection{Text extraction with the \emph{strings} command}



\section{Exercises}

What is the numeric value of ``SQLite format 3'' in base 10? Provide
both big-endian and little-endian values, as well as the  program you
used to calculate the value. 

%\input{file_type_identification}  4               % October 20

% LocalWords:  executables
