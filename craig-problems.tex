\documentclass{article}
\usepackage{enumitem}
\begin{document}
\title{File System Problem set}
\author{NPS CS3000}
\maketitle
This problem set assumes that you have read the draft ``Finding Hidden Data
in File Systems'' chapter of \emph{Technical Privacy Auditing with
  Computer Forensics Tools}

\begin{enumerate}
\item Download GEN1 (nps-2009-canon2-gen1.raw) and show a hex dump of the
  images' first sector.
\item Draw a box around hex bytes in the dump that
  correspond to the first slot in the partition table.
\item Show a hex dump of the GEN1 Boot Sector Structure.
\item Run the \texttt{mbrdecode.py} program shown in the chapter on
  GEN1 and provide the results.
\item Extend \texttt{mbrdecode.py} to decode and print all of the
  fields in the Boot Sector Structure. You will do this by:
\begin{enumerate}
  \item Determining the block number in which the BSS resides. (Hint:
    it's the first block of the partition.)
  \item Reading the block into a buffer.
  \item Using \texttt{struct.unpack} to decode the buffer. (You
  can do this in a single statement or with a series of statements as
  we do in the Listing.)
  \item Printing each value. 
\end{enumerate}


\item Extend \texttt{mbrdecode.py} to print the contents of the root
  directory.  You will do this by:
\begin{enumerate}
  \item Determining the location and length of the root directory
    from the BSS.
  \item Reading the root directory into a buffer.
  \item Creating a function that takes a buffer and prints each
    directory entry that the buffer contains. 
\end{enumerate}  

\item In order to work with files it is necessary to read the File
  allocation table. Using the information you learned in the previous
  step, provide the sector numbers where the FAT starts and stops.

\item Modify the \texttt{mbrdecode.py} program to read the
  entire FAT into an integer array. 

  Recall that sectors on a FAT volume used to hold user data are
  called \emph{clusters};  each cluster may consist of one or
  more sectors (blocks). Each file can therefore be referred to by the
  number of the cluster where the first byte of file data is
  stored. The FAT array for each cluster contains a pointer to the
  location of the next cluster or a special value indicating there are
  no more data clusters. Each file is therefore described by a linked
  list of cluster numbers, and all of the linked lists for all of the
  files are stored within the fAT.

  For the remainder of this problem set we will be working with two
  files. File \#517 is the
  directory \verb|\DCIM\100CANON\| while File \#1029 is the JPEG image
  \verb|IMG_0001.JPG| in that directory.

  Once you have modified \texttt{mbrdecode.py} to read the entire FAT
  into memory, modify it further to print all of the cluster numbers 
  associated with File \#517. Provide a list of the clusters.

\item Provide a list of the clusters associated with File \#1029.

\item Now that you have a list of data clusters, you can read the
  contents of the files and decode them!  Read File \#517 into a
  buffer and print its contents using the function you wrote that
  decodes file directories. Provide this output.

\item Read File \#1029 and store the results in a file on your host
  computer called \emph{FILE1029.jpg}. Show the contents of that file.

\end{enumerate}
  
Extra credit:


\begin{enumerate}
\item Write a program that starts at the root
  directory and recursively lists the name, modification date, and
  size of every file on the disk.
\item Create a virtual disk using the instructions in the chapter. Put
  a few files on the disk and delete a few of them. Read the directory
  and explain which files can be recovered with file recovery
  utilities. Write a file recovery utility and try to recover one or
  more of the deleted files.
\end{enumerate}

\end{document}
